\documentclass[a4paper]{article}
\usepackage[T1]{fontenc}
\usepackage[russian]{babel}
\usepackage[pdftex]{graphicx}
\usepackage[ruled,vlined]{algorithm2e}
\usepackage[utf8]{inputenc}
%\usepackage{cmake-build-debug/-wasysym-}
\graphicspath{{.}}
\usepackage{xcolor}
\usepackage{hyperref}
\usepackage{amsmath}
\DeclareGraphicsExtensions{.pdf,.png,.jpg}
\usepackage{geometry} % Меняем поля страницы
\newcommand{\comment}{}
\usepackage{float}
\usepackage{caption}
\usepackage{subcaption}
\begin{document}
    \begin{titlepage}
        \Large
        \begin{center}
            Санкт-Петербургский \\ Политехнический университет Петра Великого\\
            \vspace{10em}Отчет по лабораторной работе №14\\
            \vspace{2em}
            \textbf{Решение краевой задачи для ОДУ 2-ого порядка.}
        \end{center}
        \vspace{6em}
        \newbox{\lbox}
        \savebox{\lbox}{\hbox{Швачко Никита Андреевич}}
        \newlength{\maxl}
        \setlength{\maxl}{\wd\lbox}
        \hfill\parbox{10cm}{
            \hspace*{2cm}\hspace*{-4cm}Студент:\hfill\hbox to\maxl{Швачко Никита Андреевич\hfill}\\
            \hspace*{2cm}\hspace*{-4cm}Преподаватель:\hfill\hbox to\maxl{Козлов Константин Николаевич}\\
            \hspace*{2cm}\hspace*{-4cm}Группа:\hfill\hbox to\maxl{5030102/20001}
        }
        \vspace{\fill}
        \begin{center}
            Санкт-Петербург \\2024
        \end{center}
    \end{titlepage}


    \section{Формулировка задания и его формализация}\label{sec:----}
    Для 4 распределений:\\
    - Нормальное распределение $N(x, 0,1)$\\
    - Распределение Коши $C(x, 0,1)$\\
    - Распределение Пуассона $P(k, 10)$\\
    - Равномерное распределение $U(x,-\sqrt{3}, \sqrt{3})$\\
    1. Сгенерировать выборки размером 10,50 и 1000 элементов.
    Построить на одном рисунке гистограмму и график плотности распределения.\\
    2.
    Сгенерировать выборки размером 10,100 и 1000 элементов.
    Для каждой выборки вычислить следующие статистические характеристики положения данных:
    $\bar{x}, \operatorname{med} x, z_Q$.
    Повторить такие вычисления 1000 раз для каждой выборки и найти среднее характеристик положения и их квадратов:
    \\
    $$
    E(z)=\bar{z}
    $$


    Вычислить оценку дисперсии по формуле:

    $$
    D(z)=\overline{z^2}-\bar{z}^2
    $$


    Представить полученные данные в виде таблиц.
    \\  Пояснение

    $$
    z_Q=\frac{z_{1 / 4}+z_{3 / 4}}{2}
    $$


\end{document}
