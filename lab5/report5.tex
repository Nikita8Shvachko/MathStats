\documentclass[a4paper]{article}
\usepackage[T1]{fontenc}
\usepackage[russian]{babel}
\usepackage[pdftex]{graphicx}
\usepackage[ruled,vlined]{algorithm2e}
\usepackage[utf8]{inputenc}
\usepackage{xcolor}
\usepackage{hyperref}
\usepackage{amsmath}
\usepackage{geometry}
\usepackage{float}
\usepackage{caption}
\usepackage{subcaption}
\DeclareGraphicsExtensions{.pdf,.png,.jpg}


\begin{document}

    \begin{titlepage}
        \Large
        \begin{center}
            Санкт-Петербургский \ Политехнический университет Петра Великого\\
            \vspace{10em}Отчет по лабораторной работе №5\\
            \vspace{2em}
            \textbf{Проверка гипотезы о законе распределения генеральной совокупности. Метод хи-квадрат}
        \end{center}
        \vspace{6em}
        \hfill\parbox{10cm}{
            \hspace*{2cm}\hspace*{-4cm}Студент:\hfill Швачко Никита Андреевич\\
            \hspace*{2cm}\hspace*{-4cm}Преподаватель:\hfill Баженов Александр Николаевич\\
            \hspace*{2cm}\hspace*{-4cm}Группа:\hfill 5030102/20202
        }
        \vspace{\fill}
        \begin{center}
            Санкт-Петербург \ 2025
        \end{center}
    \end{titlepage}


    \section{Постановка задачи}
    Проверить гипотезу о нормальном распределении генеральной совокупности с использованием критерия согласия $\chi^2$.
    Для этого:
    \begin{itemize}
        \item Сгенерировать выборку из 100 элементов нормального распределения $N(0, 1)$.
        \item Оценить параметры $\mu$ и $\sigma$ методом максимального правдоподобия.
        \item Проверить гипотезу о соответствии нормальному распределению $N(\hat{\mu}, \hat{\sigma})$ с использованием
        критерия $\chi^2$ при уровне значимости $\alpha = 0.05$.
        \item Исследовать чувствительность критерия $\chi^2$, проверив равномерные выборки из 100 и 20 элементов на нормальность.
    \end{itemize}


    \section{Описание используемых методов}
    Параметры нормального распределения были оценены с использованием метода максимального правдоподобия.
    Далее выборка была разбита на 10 равных интервалов, по которым были рассчитаны наблюдаемые и ожидаемые частоты.
    Затем была вычислена статистика $\chi^2$:
    \[
        \chi^2 = \sum_{i=1}^{k} \frac{(n_i - np_i)^2}{np_i}
    \]
    Где $n_i$ — наблюдаемая частота, $p_i$ — теоретическая вероятность попадания в интервал, $n$ — объем выборки.
    Полученное значение сравнивалось с квантилем распределения $\chi^2$ с $k-1$ степенями свободы.


    \section{Результаты эксперимента}
    Результаты для нормального распределения:
    \begin{itemize}
        \item Оценка параметра $\mu$: $0.0598$
        \item Оценка параметра $\sigma$: $1.0130$
        \item Вычисленное значение $\chi^2$: $2.38$
        \item Квантиль $\chi^2$ при $\alpha = 0.05$ и $k-1=9$: $16.92$
        \item Гипотеза $H_0$ принимается.
    \end{itemize}

    \begin{table}[H]
        \centering
        \caption{Таблица вычислений критерия $\chi^2$ (равномерное распределение, 20 элементов)}
        \begin{tabular}{|c|c|c|c|}
            \hline
            Интервал         & $n_i$ & $np_i$ & $\frac{(n_i - np_i)^2}{np_i}$ \\
            \hline
            $[-1.64, -1.11]$ & 3     & 3.33   & 0.03                          \\
            $[-1.11, -0.57]$ & 6     & 3.33   & 2.13                          \\
            $[-0.57, -0.04]$ & 3     & 3.33   & 0.03                          \\
            $[-0.04, 0.50]$  & 3     & 3.33   & 0.03                          \\
            $[0.50, 1.03]$   & 2     & 3.33   & 0.53                          \\
            $[1.03, 1.57]$   & 3     & 3.33   & 0.03                          \\
            \hline
            \textbf{Итого}   &       &        & \textbf{2.78}                 \\
            \hline
        \end{tabular}
    \end{table}

    Результаты для равномерного распределения:
    \begin{itemize}
        \item Выборка из 100 элементов: $\chi^2 = 2.4$, $H_0$ принимается.
        \item Выборка из 20 элементов: $\chi^2 = 6.0$, $H_0$ принимается.
    \end{itemize}


    \section{Выводы}
    Критерий $\chi^2$ не позволил отвергнуть гипотезу о нормальности для сгенерированной выборки.
    Для равномерных выборок гипотеза о нормальности также не была отвергнута, даже при размере 100,
    что указывает на низкую чувствительность критерия $\chi^2$ при ограниченном числе интервалов и малом объеме выборки.


\end{document}