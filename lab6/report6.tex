\documentclass[a4paper]{article}
\usepackage[T1]{fontenc}
\usepackage[russian]{babel}
\usepackage[pdftex]{graphicx}
\usepackage[ruled,vlined]{algorithm2e}
\usepackage[utf8]{inputenc}
\usepackage{xcolor}
\usepackage{hyperref}
\usepackage{amsmath}
\usepackage{geometry}
\usepackage{float}
\usepackage{caption}
\usepackage{subcaption}
\DeclareGraphicsExtensions{.pdf,.png,.jpg}


\begin{document}

    \begin{titlepage}
        \Large
        \begin{center}
            Санкт-Петербургский \ Политехнический университет Петра Великого\\
            \vspace{10em}Отчет по лабораторной работе №6\\
            \vspace{2em}
            \textbf{Доверительные интервалы для параметров нормального распределения}
        \end{center}
        \vspace{6em}
        \hfill\parbox{10cm}{
            \hspace*{2cm}\hspace*{-4cm}Студент:\hfill Швачко Никита Андреевич\\
            \hspace*{2cm}\hspace*{-4cm}Преподаватель:\hfill Баженов Александр Николаевич\\
            \hspace*{2cm}\hspace*{-4cm}Группа:\hfill 5030102/20202
        }
        \vspace{\fill}
        \begin{center}
            Санкт-Петербург \ 2025
        \end{center}
    \end{titlepage}


    \section{Постановка задачи}

    Для выборок мощностью $n=20$ и $n=100$ требуется:

    \begin{enumerate}
        \item Построить доверительные интервалы для параметров:
        \begin{itemize}
            \item нормального распределения;
            \item произвольного распределения, используя асимптотический подход.
        \end{itemize}
        \item Представить результаты в виде таблиц.
    \end{enumerate}


    \section{Результаты эксперимента}

    \subsection{Доверительные интервалы для параметров нормального распределения}

    \begin{table}[H]
        \centering
        \begin{tabular}{|c|c|c|}
            \hline
            $n=20$ & $m$ & $\sigma$ \\
            \hline
            & $0.16 < m < 0.98$ & $0.66 < \sigma < 1.27$ \\
            \hline
            $n=100$ & $m$ & $\sigma$ \\
            \hline
            & $-0.17 < m < 0.26$ & $0.93 < \sigma < 1.24$ \\
            \hline
        \end{tabular}
        \caption{Доверительные интервалы для параметров нормального распределения}
    \end{table}

    \begin{table}[H]
        \centering
        \begin{tabular}{|c|c|c|}
            \hline
            $n=20$ & $m$ (твин) & $\sigma$ (твин) \\
            \hline
            & $[[0.16, 0.16], [0.98, 0.98]]$ & $[[0.66, 0.66], [1.27, 1.27]]$ \\
            \hline
            $n=100$ & $m$ (твин) & $\sigma$ (твин) \\
            \hline
            & $[[-0.17, -0.17], [0.26, 0.26]]$ & $[[0.93, 0.93], [1.24, 1.24]]$ \\
            \hline
        \end{tabular}
        \caption{Твины для параметров нормального распределения}
    \end{table}

    \subsection{Доверительные интервалы для параметров произвольного распределения. Асимптотический подход}

    \begin{table}[H]
        \centering
        \begin{tabular}{|c|c|c|}
            \hline
            $n=20$ & $m$ & $\sigma$ \\
            \hline
            & $0.19 < m < 0.95$ & $0.66 < \sigma < 1.27$ \\
            \hline
            $n=100$ & $m$ & $\sigma$ \\
            \hline
            & $-0.16 < m < 0.25$ & $0.93 < \sigma < 1.24$ \\
            \hline
        \end{tabular}
        \caption{Доверительные интервалы для параметров произвольного распределения. Асимптотический подход}
    \end{table}

    \begin{table}[H]
        \centering
        \begin{tabular}{|c|c|c|}
            \hline
            $n=20$ & $m$ (твин) & $\sigma$ (твин) \\
            \hline
            & $[[0.19, 0.19], [0.95, 0.95]]$ & $[[0.66, 0.66], [1.27, 1.27]]$ \\
            \hline
            $n=100$ & $m$ (твин) & $\sigma$ (твин) \\
            \hline
            & $[[-0.16, -0.16], [0.25, 0.25]]$ & $[[0.93, 0.93], [1.24, 1.24]]$ \\
            \hline
        \end{tabular}
        \caption{Твины для параметров произвольного распределения. Асимптотический подход}
    \end{table}


    \section{Выводы}
    В ходе выполнения лабораторной работы были построены доверительные интервалы для параметров распределения на основе выборок объёмом $n=20$ и $n=100$.

    \begin{itemize}
        \item Для нормального распределения интервалы были рассчитаны с использованием точных методов.
        \item Для произвольного распределения применён асимптотический подход, базирующийся на центральной предельной теореме.
        \item Как и ожидалось, при увеличении объёма выборки интервалы становятся уже, что указывает на повышение точности оценки параметров.
        \item Результаты подтверждают теоретические ожидания и демонстрируют корректность применённых статистических методов.
    \end{itemize}

\end{document}